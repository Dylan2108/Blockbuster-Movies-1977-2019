\documentclass{beamer}
\usetheme{Madrid}
\usecolortheme{default}

\usepackage[utf8]{inputenc}
\usepackage[spanish]{babel}
\usepackage{booktabs}
\usepackage{graphicx}

\title{Análisis Estadístico de Películas Blockbuster}
\subtitle{1977 -- 2019}
\author{
Ronald Provance Valladares \\ 
Agustin Alberto Carbajal Romero \\ 
Dylan Ramsés Cabrera Morales
}
\institute{Universidad de La Habana | MATCOM}
\date{}

\begin{document}

\begin{frame}
\titlepage
\end{frame}

%--------------------------------

\begin{frame}{Objetivos y Motivación}
\begin{itemize}
\item La industria cinematográfica mueve miles de millones anualmente
\item Comprender qué impulsa el éxito es clave para productores e inversores
\end{itemize}

\textbf{Preguntas de investigación:}
\begin{itemize}
\item ¿Qué factores influyen en la recaudación mundial?
\item ¿Existen clusters de películas similares?
\item ¿Ha cambiado la recaudación promedio con el tiempo?
\end{itemize}
\end{frame}

%--------------------------------

\begin{frame}{Descripción del Dataset}
\begin{itemize}
\item Fuente: Kaggle
\item Período: 1977--2019
\item Top 10 películas por año
\item Total: 430 películas
\end{itemize}

\textbf{Variables principales:}
\begin{itemize}
\item Presupuesto: \$15M -- \$365M
\item Recaudación
\item Rating IMDb (1--10)
\item Género
\item Clasificación MPAA
\end{itemize}
\end{frame}

%--------------------------------

\begin{frame}{Estadísticos Descriptivos}

\begin{tabular}{lccc}
\toprule
Variable & Media & Mediana & Desv. Est \\
\midrule
Presupuesto & \$108.5M & \$100.0M & \$67.8M \\
Recaudación & \$698.0M & \$654.0M & \$425.0M \\
Rating IMDb & 6.87 & 6.90 & 0.95 \\
ROI & 6.43 & 5.52 & 6.21 \\
\bottomrule
\end{tabular}

\vspace{0.3cm}

\textbf{Hallazgos:}
\begin{itemize}
\item Alta variabilidad en presupuestos y recaudación
\item Ratings generalmente altos
\item ROI muy variable
\end{itemize}
\end{frame}

%--------------------------------

\begin{frame}{Patrones Exploratorios}

\textbf{Géneros más comunes:}
\begin{itemize}
\item Acción (32\%)
\item Aventura (28\%)
\item Sci-Fi (18\%)
\item Drama (12\%)
\end{itemize}

\textbf{Clasificación MPAA:}
\begin{itemize}
\item PG-13: 45\%
\item R: 38\%
\item PG: 12\%
\item G: 5\%
\end{itemize}

\textbf{Conclusión clave:}  
Películas PG-13 tienden a recaudar más.

\end{frame}

%--------------------------------

\begin{frame}{Pregunta 1: Regresión Lineal Múltiple}

Se estimó un modelo con:

\begin{itemize}
\item Presupuesto
\item Rating IMDb
\item Género (dummies)
\item Clasificación MPAA (dummies)
\end{itemize}

\textbf{Tratamiento:}
\begin{itemize}
\item Variables continuas estandarizadas
\item Variables categóricas codificadas
\item Variables no significativas eliminadas
\end{itemize}
\end{frame}

%--------------------------------

\begin{frame}{Validación del Modelo}

\begin{itemize}
\item Media residuos $\approx 0$
\item Durbin-Watson $\approx 1.44$
\item Heterocedasticidad corregida (HC3)
\item No multicolinealidad (Cond.\# = 4.56)
\item Normalidad confirmada
\end{itemize}
\end{frame}

%--------------------------------

\begin{frame}{Resultados de Regresión}

\textbf{Rendimiento:}
\begin{itemize}
\item $R^2 = 0.697$
\item F-statistic $< 0.001$
\end{itemize}

\textbf{Factores importantes:}
\begin{itemize}
\item Presupuesto: +0.63\% por 1\%
\item Rating IMDb: +13\% por punto
\item Animación: +31\%
\item PG-13: +32\%
\end{itemize}
\end{frame}

%--------------------------------

\begin{frame}{Conclusiones de Regresión}

\begin{itemize}
\item Presupuesto es el factor más determinante
\item Rating influye significativamente
\item Clasificación PG-13 amplía audiencia
\item Animación destaca en recaudación
\end{itemize}

\textbf{Éxito = interacción de múltiples factores}
\end{frame}

%--------------------------------

\begin{frame}{Pregunta 2: K-Means Clustering}

\begin{itemize}
\item Algoritmo no supervisado
\item Variables:
\begin{itemize}
\item Presupuesto
\item Rating IMDb
\item Género
\end{itemize}
\item Estandarización + One-hot encoding
\end{itemize}
\end{frame}

%--------------------------------

\begin{frame}{Selección de K}

\begin{tabular}{lc}
\toprule
Métrica & Valor \\
\midrule
Silhouette & 0.52 \\
Davies-Bouldin & 0.89 \\
Calinski-Harabasz & 185.3 \\
\bottomrule
\end{tabular}

\vspace{0.3cm}

\textbf{K óptimo: 4 clusters}
\end{frame}

%--------------------------------

\begin{frame}{Clusters Identificados}

\textbf{Cluster 0 – Blockbusters Probados}
\begin{itemize}
\item Presupuesto alto, rating alto
\end{itemize}

\textbf{Cluster 1 – Apuestas Riesgosas}
\begin{itemize}
\item Presupuesto muy alto, rating medio
\end{itemize}

\textbf{Cluster 2 – Joyas Ocultas}
\begin{itemize}
\item Presupuesto moderado, rating alto
\end{itemize}

\textbf{Cluster 3 – Estándar Masivo}
\begin{itemize}
\item Presupuesto y rating promedio
\end{itemize}
\end{frame}

%--------------------------------

\begin{frame}{Hallazgos Clustering}

\begin{itemize}
\item Presupuesto no garantiza éxito crítico
\item Múltiples modelos de negocio coexisten
\item Acción y Sci-Fi requieren mayores presupuestos
\item Drama logra éxito con menos inversión
\end{itemize}
\end{frame}

%--------------------------------

\begin{frame}{Pregunta 3: ANOVA}

\textbf{Hipótesis:}

\begin{itemize}
\item $H_0$: Recaudación igual en todas las décadas
\item $H_1$: Alguna década difiere
\end{itemize}

\textbf{Supuestos verificados:}
\begin{itemize}
\item Normalidad
\item Homogeneidad de varianzas
\end{itemize}
\end{frame}

%--------------------------------

\begin{frame}{Resultados ANOVA}

\begin{tabular}{lc}
\toprule
Métrica & Valor \\
\midrule
F & 5.234 \\
p-value & 0.0012 \\
$\eta^2$ & 0.08 \\
\bottomrule
\end{tabular}

\vspace{0.3cm}

\textbf{Decisión:} Rechaza $H_0$
\end{frame}

%--------------------------------

\begin{frame}{Recaudación por Década}

\begin{itemize}
\item 1970s: \$550.2M
\item 1980s: \$618.5M
\item 1990s: \$701.3M
\item 2000s: \$742.8M
\item 2010s: \$821.5M
\end{itemize}

Incremento del 49\% desde 1970s
\end{frame}

%--------------------------------

\begin{frame}{Interpretación ANOVA}

\begin{itemize}
\item Diferencias estadísticamente significativas
\item Década explica 8\% de la variación
\item Otros factores también influyen fuertemente
\end{itemize}

\textbf{Tendencia creciente clara}
\end{frame}

%--------------------------------

\begin{frame}{Limitaciones}

\begin{itemize}
\item Solo Top-10 por año (sesgo)
\item Variables limitadas
\item Período específico
\end{itemize}
\end{frame}

%--------------------------------

\begin{frame}{Conclusiones Finales}

\begin{itemize}
\item Presupuesto importa, pero no lo es todo
\item Múltiples caminos al éxito
\item Calidad predice recaudación
\item Industria usa estrategias diferenciadas
\end{itemize}

\vspace{0.3cm}

\textbf{El éxito no es una fórmula única}
\end{frame}

%--------------------------------

\begin{frame}{Cierre}

Datos: Kaggle \\
Técnicas: Regresión, Clustering, ANOVA \\
Período: 1977--2019 \\
430 películas analizadas

\end{frame}

\end{document}
